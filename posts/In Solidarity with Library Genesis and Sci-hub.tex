
% xelatex xeCJK
\documentclass[11pt,a4paper]{article}
\usepackage[SlantFont,BoldFont]{xeCJK}
\usepackage{fontspec}
\setmainfont{Times New Roman}
\setCJKmainfont{SimSun}
\setCJKmonofont{仿宋_GB2312}
\usepackage{indentfirst}
\parindent 2em					%开头空两格
\linespread{1.2} 				%行距 倍


\usepackage{geometry}
\geometry{a4paper,left=2.5cm,right=2.5cm,top=2.5cm,bottom=2.5cm}
\usepackage[breaklinks,colorlinks,linkcolor=black,citecolor=blue,urlcolor=blue]{hyperref}
\usepackage{bookmark}

\title{In Solidarity with Library Genesis and Sci-hub}

\begin{document}
\maketitle

http://libgen.io

http://sci-hub.io

In Antoine de Saint Exupéry's tale the Little Prince meets a businessman
who accumulates stars with the sole purpose of being able to buy more
stars. The Little Prince is perplexed. He owns only a flower, which he
waters every day. Three volcanoes, which he cleans every week. {"It is
of some use to my volcanoes, and it is of some use to my flower, that I
own them,"} he says, {"but you are of no use to the stars that you
own"}.

There are many businessmen who own knowledge today. Consider Elsevier,
the largest scholarly publisher, whose 37\% profit
margin\textsuperscript{\protect\hyperlink{fn-1}{1}} stands in sharp
contrast to the rising fees, expanding student loan debt and
poverty-level wages for adjunct faculty. Elsevier owns some of the
largest databases of academic material, which are licensed at prices so
scandalously high that even Harvard, the richest university of the
global north, has complained that it cannot afford them any longer.
Robert Darnton, the past director of Harvard Library, says {"We faculty
do the research, write the papers, referee papers by other researchers,
serve on editorial boards, all of it for free \ldots{} and then we buy
back the results of our labour at outrageous
prices."}\textsuperscript{\protect\hyperlink{fn-2}{2}} For all the work
supported by public money benefiting scholarly publishers, particularly
the peer review that grounds their legitimacy, journal articles are
priced such that they prohibit access to science to many academics - and
all non-academics - across the world, and render it a token of
privilege.\textsuperscript{\protect\hyperlink{fn-3}{3}}

Elsevier has recently filed a copyright infringement suit in New York
against Science Hub and Library Genesis claiming millions of dollars in
damages.\textsuperscript{\protect\hyperlink{fn-4}{4}} This has come as a
big blow, not just to the administrators of the websites but also to
thousands of researchers around the world for whom these sites are the
only viable source of academic materials. The social media, mailing
lists and IRC channels have been filled with their distress messages,
desperately seeking articles and publications.

Even as the New York District Court was delivering its injunction, news
came of the entire editorial board of highly-esteemed journal Lingua
handing in their collective resignation, citing as their reason the
refusal by Elsevier to go open access and give up on the high fees it
charges to authors and their academic institutions. As we write these
lines, a petition is doing the rounds demanding that Taylor \& Francis
doesn't shut down Ashgate\textsuperscript{\protect\hyperlink{fn-5}{5}},
a formerly independent humanities publisher that it acquired earlier in
2015. It is threatened to go the way of other small publishers that are
being rolled over by the growing monopoly and concentration in the
publishing market. These are just some of the signs that the system is
broken. It devalues us, authors, editors and readers alike. It parasites
on our labor, it thwarts our service to the public, it denies us
access\textsuperscript{\protect\hyperlink{fn-6}{6}}.

We have the means and methods to make knowledge accessible to everyone,
with no economic barrier to access and at a much lower cost to society.
But closed access's monopoly over academic publishing, its spectacular
profits and its central role in the allocation of academic prestige
trump the public interest. Commercial publishers effectively impede open
access, criminalize us, prosecute our heroes and heroines, and destroy
our libraries, again and again. Before Science Hub and Library Genesis
there was Library.nu or Gigapedia; before Gigapedia there was textz.com;
before textz.com there was little; and before there was little there was
nothing. That's what they want: to reduce most of us back to nothing.
And they have the full support of the courts and law to do exactly
that.\textsuperscript{\protect\hyperlink{fn-7}{7}}

In Elsevier's case against Sci-Hub and Library Genesis, the judge said:
{"simply making copyrighted content available for free via a foreign
website, disserves the public
interest"}\textsuperscript{\protect\hyperlink{fn-8}{8}}. Alexandra
Elbakyan's original plea put the stakes much higher: {"If Elsevier
manages to shut down our projects or force them into the darknet, that
will demonstrate an important idea: that the public does not have the
right to knowledge."}

We demonstrate daily, and on a massive scale, that the system is broken.
We share our writing secretly behind the backs of our publishers,
circumvent paywalls to access articles and publications, digitize and
upload books to libraries. This is the other side of 37\% profit
margins: our knowledge commons grows in the fault lines of a broken
system. We are all custodians of knowledge, custodians of the same
infrastructures that we depend on for producing knowledge, custodians of
our fertile but fragile commons. To be a custodian is, de facto, to
download, to share, to read, to write, to review, to edit, to digitize,
to archive, to maintain libraries, to make them accessible. It is to be
of use to, not to make property of, our knowledge commons.

More than seven years ago Aaron Swartz, who spared no risk in standing
up for what we here urge you to stand up for too, wrote: {"We need to
take information, wherever it is stored, make our copies and share them
with the world. We need to take stuff that's out of copyright and add it
to the archive. We need to buy secret databases and put them on the Web.
We need to download scientific journals and upload them to file sharing
networks. We need to fight for Guerilla Open Access. With enough of us,
around the world, we'll not just send a strong message opposing the
privatization of knowledge --- we'll make it a thing of the past. Will
you join us?"}\textsuperscript{\protect\hyperlink{fn-9}{9}}

We find ourselves at a decisive moment. This is the time to recognize
that the very existence of our massive knowledge commons is an act of
collective civil disobedience. It is the time to emerge from hiding and
put our names behind this act of resistance. You may feel isolated, but
there are many of us. The anger, desperation and fear of losing our
library infrastructures, voiced across the internet, tell us that. This
is the time for us custodians, being dogs, humans or cyborgs, with our
names, nicknames and pseudonyms, to raise our voices.

Share this letter - read it in public - leave it in the printer. Share
your writing - digitize a book - upload your files. Don't let our
knowledge be crushed. Care for the libraries - care for the metadata -
care for the backup. Water the flowers - clean the volcanoes.

30 November 2015

Dušan Barok, Josephine Berry, Bodó Balázs, Sean Dockray, Kenneth
Goldsmith, Anthony Iles, Lawrence Liang, Sebastian Lütgert, Pauline van
Mourik Broekman, Marcell Mars, spideralex, Tomislav Medak, Dubravka
Sekulić, Femke Snelting...

\begin{center}\rule{0.5\linewidth}{\linethickness}\end{center}

\begin{enumerate}
\item
  \protect\hypertarget{fn-1}{}{Larivière, Vincent, Stefanie Haustein,
  and Philippe Mongeon.
  ``\href{http://journals.plos.org/plosone/article?id=10.1371/journal.pone.0127502}{The
  Oligopoly of Academic Publishers in the Digital Era.}'' PLoS ONE 10,
  no. 6 (June 10, 2015): e0127502. doi:10.1371/journal.pone.0127502.,\\
  ``\href{http://svpow.com/2012/01/13/the-obscene-profits-of-commercial-scholarly-publishers/}{The
  Obscene Profits of Commercial Scholarly Publishers.}'' svpow.com.
  Accessed November 30, 2015. ~\protect\hyperlink{fnref-1}{}}
\item
  \protect\hypertarget{fn-2}{}{Sample, Ian.
  ``\href{http://www.theguardian.com/science/2012/apr/24/harvard-university-journal-publishers-prices}{Harvard
  University Says It Can't Afford Journal Publishers' Prices.}'' The
  Guardian, April 24, 2012, sec. Science. theguardian.com.
  ~\protect\hyperlink{fnref-2}{}}
\item
  \protect\hypertarget{fn-3}{}{``\href{http://www.aljazeera.com/indepth/opinion/2012/10/20121017558785551.html}{Academic
  Paywalls Mean Publish and Perish - Al Jazeera English.}'' Accessed
  November 30, 2015. aljazeera.com. ~\protect\hyperlink{fnref-3}{}}
\item
  \protect\hypertarget{fn-4}{}{``\href{https://torrentfreak.com/sci-hub-tears-down-academias-illegal-copyright-paywalls-150627/}{Sci-Hub
  Tears Down Academia's `Illegal' Copyright Paywalls.}'' TorrentFreak.
  Accessed November 30, 2015. torrentfreak.com.
  ~\protect\hyperlink{fnref-4}{}}
\item
  \protect\hypertarget{fn-5}{}{``\href{https://www.change.org/p/save-ashgate-publishing}{Save
  Ashgate Publishing.}'' Change.org. Accessed November 30, 2015.
  change.org. ~\protect\hyperlink{fnref-5}{}}
\item
  \protect\hypertarget{fn-6}{}{``\href{http://thecostofknowledge.com/}{The
  Cost of Knowledge.}'' Accessed November 30, 2015.
  thecostofknowledge.com. ~\protect\hyperlink{fnref-6}{}}
\item
  \protect\hypertarget{fn-7}{}{{In fact, with the TPP and TTIP being
  rushed through the legislative process, no domain registrar, ISP
  provider, host or human rights organization will be able to prevent
  copyright industries and courts from criminalizing and shutting down
  websites "expeditiously". ~\protect\hyperlink{fnref-7}{}}}
\item
  \protect\hypertarget{fn-8}{}{``\href{https://torrentfreak.com/court-orders-shutdown-of-libgen-bookfi-and-sci-hub-151102/}{Court
  Orders Shutdown of Libgen, Bookfi and Sci-Hub.}'' TorrentFreak.
  Accessed November 30, 2015. torrentfreak.com.
  ~\protect\hyperlink{fnref-8}{}}
\item
  \protect\hypertarget{fn-9}{}{``\href{https://archive.org/stream/GuerillaOpenAccessManifesto/Goamjuly2008_djvu.txt}{Guerilla
  Open Access Manifesto.}'' Internet Archive. Accessed November 30,
  2015. archive.org. ~\protect\hyperlink{fnref-9}{}}
\end{enumerate}

\end{document}
